\documentclass[neutrino.tex]{subfiles}

\begin{document}

\section{I neutrini nel Modello Standard}
  \label{sec:il-neutrino-nel-modello-standard}
  Il Modello Standard è una teoria che classifica tutte le particelle
  fondamentali e descrive le leggi che regolano le loro interazioni.
  Secondo questa teoria i neutrini appartengono alla categoria dei
  leptoni: particelle con spin $1/2$, non soggette all'interazione forte,
  organizzate in 3 generazioni.\\
  \begin{equation*}
      \begin{pmatrix}
        \nu_e \\ e^{\text{-}}
      \end{pmatrix}
      \begin{pmatrix}
        \nu_{\mu} \\ \mu^{\text{-}}
      \end{pmatrix}
      \begin{pmatrix}
        \nu_{\tau} \\ \tau^{\text{-}}
      \end{pmatrix}
  \end{equation*}\\
  I neutrini inoltre, al contrario degli altri leptoni, sono caratterizzati
  da massa e carica elettrica nulle, possono quindi interagire solo tramite
  interazione debole.\\



\end{document}